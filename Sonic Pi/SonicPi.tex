\documentclass[11pt, oneside]{amsart}
\usepackage{geometry}
\usepackage{graphicx}
\usepackage{amssymb}
\geometry{letterpaper}

\title{Using Airplay on the raspberry Pi}
\author{Aaron Walker - Raspberry Pu Club}

\begin{document}
\maketitle
  Sonic PI is a program that allows you to make music. The program is installed on the raspberry pi by default. For this example we will need headphones.
  \begin{enumerate}
    \item
      We first need to set the audio to go through the audio port on the pi, rather than the HDMI Port
      \begin{verbatim}
        raspi-config
      \end{verbatim}
    \item Configure the sound to go through the 3.5 mm jack
    \item Launch sonic Pi, the icon should be on the desktop
    \item Test out the following examples - run the program and listen to what is produced.
    \begin{verbatim}
      loop do
	      sample :perc_bell, rate: (rrand 0.125, 1.5)
	      sleep rrand(0,2)
      end
    \end{verbatim}
    Then try this:
    \begin{verbatim}
      with_fx :reverb, mix: 0.5 do
        loop do
          s = synth [:bnoise, :cnoise, :gnoise].choose, amp: rrand(0.5, 1.5),⏎
      attack: rrand(0, 4), sustain: rrand(0, 2), release: rrand(1, 3), ⏎
      cutoff_slide: rrand(0, 3), cutoff: rrand(60, 80), pan: rrand(-1, 1), ⏎ pan_slide: 1, amp: rrand(0.5, 1)
          control s, pan: rrand(-1, 1), cutoff: rrand(60, 115)
          sleep rrand(2, 3)
        end
      end
    \end{verbatim}
  \end{enumerate}
  The Dr Who theme tune was made using a synth. Can you try and make a program that can reproduce it?
\end{document}
